Our approach followed an automatic mechanism to extract, refine and analyze source codes without human intervention. By utilizing this approach we are able to have a big data set helping us to achieve more precise result. Certainly we want to have equal number of Java projects to be able comparing evolution of these two different language. SonarQube is capable of measuring metrics regardless of target language. That study would certainly reveal hidden corners that we did not catch by studying only JavaScript projects and if the outcome shows similarities, making decision for stakeholders, project owners and developers would be easier to have best practices from Java which established for more than a decade in an enterprise level applications.
JavaScript lacks standard language specification for defining classes. Classes are used to achieve re-usability in JavaScript. We developed a tool for finding these classes using static analysis. For another study we extract object oriented metrics from well-known JavaScript such as McCabe cyclomatic complexity, coupling between objects \cite{Briand-Coupling} and cohesion\cite{Briand-Cohesion}. These metrics can help us to better understand JavaScript projects and also we can compare these metrics with values extracted from Java projects correspondingly. 
We plan to extensively study differences and similarities between Java and JavaScript project to gain deeper insight into these two different world. To achieve our goal we have been considering every corner cases we would face in our study like lacking of explicit class definition in JavaScript and different style of defining namespaces in JavaScript. Namespaces are the only way JavaScript developers rely to achieve higher degree of granularity.