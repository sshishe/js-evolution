
JavaScript is object-oriented to its core, with powerful, flexible high level programming capabilities. This language also supports functional and imperative programming styles. It is ubiquitous, it is fast and getting faster as compare to other web programming languages. Developers can craft it manually or they can target it by compiling from another language \footnote{TypeScript: http://www.typescriptlang.org/Tutorial} \footnote{CoffeeScript: http://coffeescript.org/}. It has been few years since JavaScript \footnote{NodeJS: http://nodejs.org/} is competing with other server side languages like (PHP, Ruby and etc.)

Source code analysis in object oriented and generally statically typed languages has been the interest of researchers for decades. However measuring object oriented metrics in JavaScript is scarce \cite{Richards:2010:ADB:1809028.1806598} \cite{6320536}.

We decide to examine the evolution of five JavaScript server-side packages known as Node Packages and five Java projects to compare how might the similarities and differences on how these two popular programming languages can  with their own features affect evolution of software systems.  Table \ref{eval_table} lists the projects and preliminary results of each project.