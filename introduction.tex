
Source code analysis in object oriented languages has been the interest of researchers for decades and, they are generally interested in statically typed languages. There are different branches of research that came to life because of this interest. We can measure our software, check the health of our design, propose refactorings and even predict errors based on the characteristics of the source code. All of this is possible today because our knowledge of the source code is advancing.  However studies measuring object oriented metrics in JavaScript is scarce \cite{Richards:2010:ADB:1809028.1806598 , 6320536}and, there is an exponential increase of the use of JavaScript and therefore an exponential need to understand the characteristics of this language and how these projects evolves.
\todo{put the citations here}

JavaScript is object-oriented to its core, with powerful, flexible high level programming capabilities. This language also supports functional and imperative programming styles. It is ubiquitous, it is fast and getting faster comparing to other web programming languages. Developers can craft it manually or they can target it by compiling from another language \footnote{TypeScript: http://www.typescriptlang.org/Tutorial} \footnote{CoffeeScript: http://coffeescript.org/}. It has been few years since JavaScript \footnote{NodeJS: http://nodejs.org/} is competing with other server side languages like (PHP, Ruby and etc.)

In this paper we present an empirical study on the evolution of fifteen open source projects that are implemented entirely in JavaScript. These projects are well know in the JavaScript community and they are frequently adopted in many projects through the web. We selected projects with an average of 5 years of development history, which gives us the possibility to analyze over 57 years of evolution in more than 1065 releases. 

First, we analyze the growth of the projects during its life-cycle. We measure the lines of code, number of commented lines, the number of directories, the number of functions,statements and complexity. 

Second, we analyze the characteristics of the project in terms of time to ship new releases and how frequently public APIs change. This is an important factor to understand, due to the development nature of the language, which is highly modular and fast passed. JavaScript libraries are created, shared and combined with other JavaScripts libraries with a hight frequency and often they are modified to fit a specific purpose. This is a reflexion of the dynamic environment that the language is mostly used. We examine the release density of five projects which has different release policies. 

Third, we conduct a quantitative study on how JavaScript community gets involved with the project development. We discuss the relation between the number of developers in different periods of the project and the number of issues that are reported and fixed. Then we analyze the average time that it takes to fix an issue over the different JavaScript projects and explore the how the engagement and availability of developers combined impacts the evolution of the project.

The rest of the paper is organized as follows: In \ref{sec:applications} we explain in details the projects selected to conduct this study, and how much big our dataset is in terms of years of development, commits and number of releases. \todo{modify with the other sections when is done}