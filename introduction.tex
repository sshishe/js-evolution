
JavaScript is object-oriented to its core, with powerful, flexible high level programming capabilities. This language also supports functional and imperative programming styles. It is ubiquitous, it is fast and getting faster as compare to other web programming languages. Developers can craft it manually or they can target it by compiling from another language \footnote{TypeScript: http://www.typescriptlang.org/Tutorial} \footnote{CoffeeScript: http://coffeescript.org/}. It has been few years since JavaScript \footnote{NodeJS: http://nodejs.org/} is competing with other server side languages like (PHP, Ruby and etc.)

Source code analysis in object oriented and generally statically typed languages has been the interest of researchers for decades. However measuring object oriented metrics in JavaScript is scarce \cite{Richards:2010:ADB:1809028.1806598} \cite{6320536}.


In this paper we present an empirical study on fifteen open source projects that are implemented entirely in JavaScript to study their evolution. We examined several metrics on over numerous releases that we collect in our dataset. We examined release density of five projects with different behavior in their release policy. We also studied how much modular are JavaScript projects.
 In terms of community contribution we constitute a quantitative study on how JavaScript community involve in project development. Then we discuss the correlation that exist between number of developers in different periods of project and the number of issues reported and get fixed. The average time it takes to fix an issue in different JavaScript projects is an indication of willingness and availability of developers combined with the number of participant.

The rest of the paper is organized as follows: In \ref{sec:applications} we explain what projects we are selected and how much big our dataset is in terms of years of development, commits and number of releases.

