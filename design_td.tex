\documentclass[conference]{IEEEtran}

\usepackage{amssymb,amsmath}
\usepackage{wrapfig}
\usepackage{multirow}
\usepackage{graphicx}
\usepackage{algorithm}
\usepackage{algorithmic}
\usepackage{times}
\usepackage{cite}
\usepackage{url}
\usepackage{booktabs}
\usepackage{subfigure}
\usepackage{fancybox}
\usepackage{color}
\usepackage{float}
\usepackage{array}
\usepackage{subfigure}
\usepackage{balance}
\usepackage{epstopdf}
\usepackage{array}
\usepackage{listings}
\usepackage{color}
\definecolor{lightgray}{rgb}{.9,.9,.9}
\definecolor{darkgray}{rgb}{.4,.4,.4}
\definecolor{purple}{rgb}{0.65, 0.12, 0.82}
\usepackage{array,graphicx}
\usepackage{booktabs}
\usepackage{pifont}
\usepackage{tablefootnote}
\usepackage{footnote}
\usepackage{multirow}


\usepackage{threeparttable}

\newcommand{\shahriar}[1]{\textcolor{red}{{\it [Shahriar: #1]}}}
\newcommand{\everton}[1]{\textcolor{blue}{{\it [Everton: #1]}}}
\newcommand{\todo}[1]{\colorbox{yellow}{\textbf{[#1]}}}



\lstdefinelanguage{JavaScript}{
	keywords={typeof, new, true, false, catch, function, return, null, catch, switch, var, if, in, while, do, else, case, break},
	keywordstyle=\color{blue}\bfseries,
	ndkeywords={class, export, boolean, throw, implements, import, this},
	ndkeywordstyle=\color{darkgray}\bfseries,
	identifierstyle=\color{black},
	sensitive=false,
	comment=[l]{//},
	morecomment=[s]{/*}{*/},
	commentstyle=\color{purple}\ttfamily,
	stringstyle=\color{red}\ttfamily,
	morestring=[b]',
	morestring=[b]"
}

\lstset{
	language=JavaScript,
	backgroundcolor=\color{lightgray},
	extendedchars=true,
	basicstyle=\footnotesize\ttfamily,
	showstringspaces=false,
	showspaces=false,
	numbers=left,
	numberstyle=\footnotesize,
	numbersep=5pt,
	tabsize=2,
	breaklines=true,
	showtabs=false,
	captionpos=b
}

\newcommand{\header}[1]{\par\medskip\noindent\textbf{#1.}}
\newcommand{\crawljax}{\textsc{Crawljax}\xspace}

\newcommand{\duptype}[1]{
	\vspace{6pt}
	\noindent
	\ovalbox{
		\begin{minipage}{8cm}
			#1
		\end{minipage}
	}
	\vspace{4pt}   
}

\newcommand*{\escape}[1]{\texttt{\textbackslash#1}}

\newcommand{\conclusionbox}[1]{%
	\vspace{2mm}
	\framebox[0.45\textwidth][c]{%
		\parbox[b]{0.42\textwidth}{%
			{\it #1}
		}
	}
	\vspace{2mm}
}

\lstdefinelanguage{JavaScript}{
	keywords={typeof, new, true, false, catch, function, return, null, catch, switch, var, if, in, while, do, else, case, break},
	keywordstyle=\color{blue}\bfseries,
	ndkeywords={class, export, boolean, throw, implements, import, this},
	ndkeywordstyle=\color{darkgray}\bfseries,
	identifierstyle=\color{black},
	sensitive=false,
	comment=[l]{//},
	morecomment=[s]{/*}{*/},
	commentstyle=\color{purple}\ttfamily,
	stringstyle=\color{red}\ttfamily,
	morestring=[b]',
	morestring=[b]"
}

\lstset{
	language=JavaScript,
	backgroundcolor=\color{lightgray},
	extendedchars=true,
	basicstyle=\footnotesize\ttfamily,
	showstringspaces=false,
	showspaces=false,
	numbers=left,
	numberstyle=\footnotesize,
	numbersep=5pt,
	tabsize=2,
	breaklines=true,
	showtabs=false,
	captionpos=b
}


First, we analyze the growth of the projects during its life-cycle. We measure the lines of code, number of commented lines, the number of directories, the number of functions,statements and complexity. 

Second, we analyze the characteristics of the project in terms of time to ship new releases and how frequently public APIs change. This is an important factor to understand, due to the development nature of the language, which is highly modular and fast passed. JavaScript libraries are created, shared and combined with other JavaScripts libraries with a hight frequency and often they are modified to fit a specific purpose. This is a reflexion of the dynamic environment that the language is mostly used. We examine the release density of five projects which has different release policies. 

Third, we conduct a quantitative study on how JavaScript community gets involved with the project development. We discuss the relation between the number of developers in different periods of the project and the number of issues that are reported and fixed. Then we analyze the average time that it takes to fix an issue over the different JavaScript projects and explore the how the engagement and availability of developers combined impacts the evolution of the project. 

Fourth, we inspect bad practices in the source code and anti-patterns, also know as bad smells. We quantify the different kind of faults that were found during the evolution of the projects and sort them by its severity: blocker, critical, major, minor and info. There are more than 75 rules to classify a piece of code as an fault. We found that there are some bad smells that are specific for the JavaScript language and that that are other that has the same concept applied to statically typed languages.

More specidically we want to analyze first, the growth of JavaScript projects. Second, the time to release new versions and how frequently the functions in JavaScript changes. Third, how actevelly developers from the open source community colaborates to the projects and what is the average time to fix a bug in these projects. Forth, How common it is to have bad practices and anti-patterns in JavaScript projects. In order to do that we use the collected data from 15 open source projects as described in our approach to anwser our research questions. 

\newcommand{\rqi}{\textbf{RQ1 - Are JavaScript projects growing over time ?}}
\newcommand{\rqii}{\textbf{RQ2 - How often new releases are shipped in JavaScript projects and how often changes in the public API are made ?}}
\newcommand{\rqiii}{\textbf{RQ3 - How many developers are colaborating togheter in these projects and how much time it takes to a bug be fixed in these projects ?}}
\newcommand{\rqiv}{\textbf{RQ4 - How common anti-patterns are in JavaScript projects and how severe they are ?}}

\begin{document}
\title{An Empirical Study on Evolution of Open Source JavaScript Projects}

\author{\IEEEauthorblockN{Everton da S. Maldonado and Shahriar Rostami Dovom}
	
	\IEEEauthorblockA{Department of Computer Science and Software Engineering\\Concordia University\\
		Montreal, Canada\\
		\url{everton.maldonado@gmail.com}, \url{shahriar.rostami@gmail.com}}}

\maketitle

\section{Introduction}
\label{sec:introduction}

JavaScript is object-oriented to its core, with powerful, flexible high level programming capabilities. This language also supports functional and imperative programming styles. It is ubiquitous, it is fast and getting faster as compare to other web programming languages. Developers can craft it manually or they can target it by compiling from another language \footnote{TypeScript: http://www.typescriptlang.org/Tutorial} \footnote{CoffeeScript: http://coffeescript.org/}. It has been few years since JavaScript \footnote{NodeJS: http://nodejs.org/} is competing with other server side languages like (PHP, Ruby and etc.)

Source code analysis in object oriented and generally statically typed languages has been the interest of researchers for decades. However measuring object oriented metrics in JavaScript is scarce \cite{Richards:2010:ADB:1809028.1806598} \cite{6320536}.

We decide to examine the evolution of five JavaScript server-side packages known as Node Packages and five Java projects to compare how might the similarities and differences can affect developers outcome with respect to the programming language paradigm and built-in language constructs.  Table \ref{eval_table} lists the projects and preliminary results of each project.

\section{Applications}
\label{sec:applications}
We select a long-lived projects to use in our case study based on the following criteria:
We select projects that are popular between developers and are actively developed for at least three consecutive years.
	We used projects that have their source code and issuer tracker both hosted on famous open source repository, GitHub. This makes it easier for us to collect information about collaboration and to measure popularity of the projects based on information we gather from website. Besides, GitHub provides us an API to extract data from their repositories to co-evaluate with the evolution analysis that we will run. 
	Each project should have at least 20 unique tags on Github.
	Large number of commits was also one of our criterion to select the project.

Based on Samoladas\cite{Samoladas2010SAD}, majority of open source softwares are not developed after a short time period, makes them inappropriate for evolution analysis. Our dataset consist of total 
57 years of history with 1065 releases. 

Our study conducted on projects listed as following:
\begin{itemize}
	\item \textbf{CoffeeScript} is a superset language of JavaScript and it adds syntactic sugar inspired by Python, Ruby and Haskell.
	\item \textbf{less.js} extends CSS with dynamic behavior such as variables, mixins, operations and functions. Less runs on both the server-side and client-side.
	\item \textbf{NPM} is the package manager for JavaScript either in browser or server-side.
	\item \textbf {Mongoose} is an elegant Mongodb object modeling for node.js.
	\item \textbf{Underscore} is a JavaScript library that provides a whole useful functional programming helpers without extending any built-in objects.
	\item \textbf {Node-mysql} is a node.js driver for mysql.
	\item \textbf {q} is a tool for creating and composing asynchronous promises in JavaScript.
	\item \textbf {Request} is a simplified HTTP request client for node.js.
	\item \textbf{Ember.js} is an open-source client-side JavaScript web application framework based on the model-view-controller (MVC) software architectural pattern.
	\item \textbf{Source-Map} is a library to generate and consume the source map.
	\item \textbf{Bootstrap} intuitive, and powerful mobile first front-end framework for faster and easier web development.
\end{itemize} 


\section{Approach}
\label{sec:approach}
 \begin{figure*}[thb!]
 	\caption{Approach overview}
 	\centering
 	\label{fig:approach_overview}
 	\includegraphics[width=1\textwidth]{figures/approach_overview}
 \end{figure*}

We ran our empirical study on 10 open source applications, 5 of then are written in JavaScript and the other 5 were written in Java. We present the approach overview for our study as shown in Fig.\ref{fig:approach_overview}. In this section we explain in detail how we realized the project selection, then we discuss metrics and particularities of the JavaScript language taken in consideration for this study and after that, we explain how we ran our analysis and finally we use the results to answer our research questions in section \ref{sec:rq}.

\subsection{Project Selection}

In selecting projects for our study, we used several criteria. First, we selected projects that are popular between developers and with a large number of users, since we want to analyze their evolution is important that they are active projects. Second, we used projects that has their source code and issuer tracker both hosted in GitHub. This makes easier to collect information about collaboration and measure popularity of the projects besides of that GitHub provides us an API to extract data from their repositories that will collaborate with the evolution analysis that we will run. Third we consider at this point projects with similar sizes. We will enhance this criteria with size and application domain in our final report so we can be more assertive when searching for insights about the evolution of the projects.

Regarding the lifespan of the analyzed projects we considered 5 previous full releases, regardless of the time that it took to be released. We consider a full release accordingly to the analyzed project. These releases are the ones with major changes in design and addition of new features as well, because of that we considered them as good candidates to conduct our study.  

As mentioned before we ran our analysis in 10 open source projects, 5 of them were written in JavaScript and namely they are NPM a package manager for JavaScript, Node MySQL a pure node.js JavaScript Client implementing the MySql protocol, Esprima a high performance and standard-compliant JavaScript parser, Grunt a JavaScript task runner and Node Redis a Redis client for node. Table \ref{tab:java_script_proj_details} presents hight level data of these projects.

In the same way the 5 Java projects analyzed were ElasticSearch a distributed, RESTful search engine, Gauva the Google Core Libraries for Java 6+, JodaTime is widely used as replacement for the Java date and time classes, Jsoup a Jason library for Java and finally JUnit an unit test framework for Java. Their data are presented in Table \ref{tab:java_proj_details}     

\begin{table*}[!tbh]
	\begin{center}
		\caption{JavaScript Project Details }
		\label{tab:java_script_proj_details}
		\begin{tabular}{l| c c c c c}
			\toprule
			\textbf{Project} & \textbf{\# of JS files in latest release} & \textbf{Number of Directories} & \textbf{LOC\footnote{Lines of code}} & \textbf{Number of Functions} & \textbf{Number of Statements} \\ 
			\midrule
			NPM              & 165                                       & 32                             & 9,075                                & 1,217                        & 5,329                         \\
			Node MySQL       & 140                                       & 11                             & 4,720                                & 667                          & 3,317                         \\
			Esprima          & 34                                        & 6                              & 83,385                               & 4,862                        & 29,002                        \\
			Grunt            & 31                                        & 9                              & 2,361                                & 251                          & 1,245                         \\
			Node Redis       & 18                                        & 6                              & 2,529                                & 457                          & 2,537                         \\ 
			\bottomrule
		\end{tabular}
	\end{center}
\end{table*}

\begin{table*}[!tbh]
	\begin{center}
		\caption{Java Project Details }
		\label{tab:java_proj_details}
		\begin{tabular}{l| c c c c c c}
			\toprule
			\textbf{Project} & \textbf{\# of JS files in latest release} & \textbf{Number of Directories} & \textbf{LOC\footnote{Lines of code}} & \textbf{Number of Functions} & \textbf{Classes} & \textbf{Number of Statements} \\ 
			\midrule
			ElasticSearch    & 4,050                                     & 831                            & 424,007                              & 35,762                       & 5,967            & 198,944                       \\
			Gauva            & 799                                       & 28                             & 90,401                               & 11,769                       & 1,644            & 32,698                        \\
			JodaTime         & 327                                       & 15                             & 84,855                               & 9,560                        & 473              & 50,609                        \\
			Jsoup            & 80                                        & 14                             & 13,672                               & 1,487                        & 154              & 7,980                         \\
			JUnit            & 392                                       & 73                             & 26,079                               & 3,479                        & 1,063            & 7,972                         \\ 
			\bottomrule
		\end{tabular}
	\end{center}
\end{table*}

%\begin{table*}\label{eval_table}\centering
%	\caption{Proposed experiment projects with preliminary results of most recent version of release in our dataset}
%	\begin{threeparttable}
%		\scalebox{0.7}{
%			\begin{tabular}{llccccc}
%				\toprule
%				Project & Description &  {\# of JS files in latest release} & Number of Directories & LOC \footnote{Lines of code}& Number of Functions & Number of Statements  \\
%				\addlinespace
%				\midrule
%				NPM & Package manager for JavaScript & 165 & 32 & 9,075 & 1,217 & 5,329  \\
%				\addlinespace
%				\midrule
%				Node MySQL & A pure node.js JavaScript Client implementing the MySql protocol.  & 140 & 11& 4,720 & 667 & 3,317 \\			
%				\addlinespace 	
%				\midrule
%				Esprima & A high performance, standard-compliant JavaScript parser written in JavaScript  & 34 & 6 & 83,385 & 4,862 & 29,002  \\
%				\addlinespace		
%				\midrule
%				Grunt & The JavaScript Task Runner & 31 & 9 & 2,361 & 251 & 1,245 \\
%				\addlinespace
%				\midrule
%				Node Redis & Redis client for node & 18 & 6 & 2,529 &  457 & 2,537   \\	
%				\addlinespace 
%				
%			\end{tabular}
%		}
%	\end{threeparttable}
%\end{table*}
%
%\begin{table*}\label{tab:eval_java}\centering
%	\caption{Proposed experiment projects with preliminary results of most recent version of release in our dataset}
%	\begin{threeparttable}
%		\scalebox{0.7}{
%			\begin{tabular}{llccccccc}
%				\toprule
%				Project & Description & {\# of java files in latest release} & Number of Directories & LOC & Number of Functions & classes & Number of Statements \\
%				\addlinespace
%				\midrule
%				ElasticSearch & Open Source, Distributed, RESTful Search Engine 
%				& 4,050 & 831 & 424,007 & 35,762 & 5,967 & 198,944 \\
%				\addlinespace
%				\midrule
%				Gauva & Google Core Libraries for Java 6+ & 799 & 28 & 90,401 & 11,769 & 1,644 & 32,698\\    
%				\addlinespace 
%				\midrule
%				JodaTime & Joda-Time is the widely used replacement for the Java date and time classes. & 327 & 15 & 84,855 & 9,560 & 473 & 50,609 \\
%				\addlinespace    
%				\midrule
%				Jsoup & Jason library for java & 80 & 14 & 13,672 & 1,487 & 154 & 7,980 \\
%				\addlinespace
%				\midrule
%				JUnit & Unit test framework for java & 392 & 73 & 26,079 & 3,479 & 1,063 & 7,972  \\   	
%				\addlinespace 
%			\end{tabular}
%		}
%	\end{threeparttable}
%\end{table*}

\subsection{Define set of metrics}

Lehman suggests using the number of “modules” as the best way to measure the size of a large software system \cite{Lehman1997METRICS}. However, we decided to use the number of uncommented lines of code (“uncommented LOC”) like the way Godfrey et al \cite{Godfrey2000ICMS} did the evolution study on Linux Kernel. On the other hand we measure the comment lines and the ratio of comments to lines of codes, and based on that we can infer how much developers tend to put comments within their codes. We have to consider hidden corners that can mislead results, for example descriptive comments are totally different to the lines of codes that got commented because of refactoring or changes which consider as light-weight code smells within the code.

We want to measure various aspects of the growth of these applications by having metrics such as number of files, lines of code, number of functions and statements. We also measure amount of duplications known as clones in terms of lines of codes, blocks and files. We measure the cyclomatic complexity over time which the metric is calculated as following. Whenever the control flow of a function splits, the complexity counter gets incremented by one. Each function has a minimum complexity of 1. The control flow can split by conditional statements like if/else, switch case and so on. This metric is also known as also known as McCabe metric
We use the term “source file” to mean any file whose name ends with “.js” and also we removed folders containing external libraries which is usually located at \textit{lib} or \textit{node\_modules}. 

We also measure the amount of object oriented principles that JavaScript developers use in their day to day software development. We think if we quantify the amount of reusable parts (i.e classes) in these projects, there are some valuable reasons laid down related to evolution behind these techniques. Consequently we use a project, known as JSDeodorant to find class declarations and places developers instantiate objects.
In the rest of this section we describe how developers create objects in JavaScript and how they mimic object oriented class definitions without having direct language support in the specification of language.

\noindent\textbf{Creation Types:} Following we explain different types of object creation no matter if they are built-in type or user-defined. 
%\subsubsection{Creation Types}

\medskip
\noindent\subsubsection{Array Literal Expression}
%\duptype{\textbf{Type I}: \textit{Array Literal Expression}.}

it creates a string array consisting of three creations possible in JavaScript elements and is assigned to variable “cars” using a binary operator (with two operand and equal operator). 
\medskip
\begin{lstlisting}[caption={Array literal expression},label={lst:array_literal},language=JavaScript]
var cars = ['Saab', 'Volvo', 'BMW'];
\end{lstlisting}
A JavaScript array is initialized with the given elements, except in the case where a single argument is passed to the Array constructor and that argument is a number. Note that this special case only applies to JavaScript arrays created with the Array constructor, not array literals created with the bracket syntax.
\\
%\break
%\duptype{\textbf{Type II}: \textit{Array Creation using \textbf{new} keyword}.}
\noindent\subsubsection{Array Creation using \textbf{new} keyword}

The Array constructor function with using the “New” keyword creates an array of three elements and then assigned the created object to variable “planes” using binary operator. Using the more verbose method: \textit{new Array()} instead of array literal expression does have one extra option in the parameters: if you pass a number to the constructor, you will get an array of that length. 

\medskip
\begin{lstlisting}[caption={Array constructor},label={lst:array_constructor},language=JavaScript]
var planes= new Array('Boeing', 'Airbus', 'Bombar- dier');
\end{lstlisting}

\noindent\subsubsection{Object Literal Expression}

%\duptype{\textbf{Type III}: \textit{Object Literal Expression}.}

The created object is basically singletons with variables/methods that are all public. An object literal is a comma-separated list of name-value pairs wrapped in curly braces. Object literals encapsulate data, enclosing it in a tidy package. This minimizes the use of global variables which can cause problems when combining code. If any of the syntax rules are broken, such as a missing comma or colon or curly brace, a JavaScript error will be triggered. No need to invoke constructors directly or maintain the correct order of arguments passed to functions.
\begin{lstlisting}[caption={Object literal expression},label={lst:object_literal_expression},language=JavaScript] 
var myObj = {
	myMethod: function(params) {
		// ...do something
	}
};
\end{lstlisting}

\noindent\subsubsection{Function Constructor}
%\duptype{\textbf{Type IV}: \textit{Function Constructor}.}

Listing \ref{lst:function_constructor}, shows the function constructor, there we define a function that should start with an uppercase letter by convention (to inform call sites use this function with “new” keyword). The Function constructor creates a new Function object and in JavaScript every function is actually a Function object.

Parameters are Names to be used by the function as formal parameter names. Each must be a string that corresponds to a valid JavaScript identifier or a list of such strings separated with a comma. Functions created with the Function constructor do not create closures to their creation contexts; they always are created in the global scope.

When running them, they will only be able to access their own local variables and global ones, not the ones from the scope in which the Function constructor was called.

\begin{lstlisting}[caption={Function constructor},label={lst:function_constructor},language=JavaScript] 
function Employee(name){
	this.name = name;
	this.getName = function(){
		return this.name;
	};	
};
var emp = new Employee ('John');
\end{lstlisting}
	
\section{Case Study results}
\label{sec:results}
Our goal is to understand how JavaScripts projects evolves. First we want to analyze the growth of JavaScript projects. Second, the time to release new versions and how frequently the functions in JavaScript changes. Third, how actively developers from the open source community collaborates to the projects and what is the average time to fix a bug in these projects. Fourth, How common is it to have bad practices and anti-patterns in JavaScript projects. In order to do that we use the collected data from 15 open source projects as described in our approach to answer our research questions. 

\vspace{3 mm}
\noindent{\rqi}
\vspace{3 mm}

\noindent{\textbf{Motivation:}} Growth is an important aspect of evolution. Even more to open source projects. Growth can mean that the project is in use, that it is changing over time and that it may be adapting to the needs of users. However, it is important to understand how this growing is happening. If a project grows without any planning or if it is becoming too complex that users and developers abandon it. In this case growth is a bad sign. Therefore we conduct our study to understand if the projects are growing in a maintainable manner and if the developers are building the software to the future.   

\vspace{1 mm}
\noindent{\textbf{Approach:}} First we measure the lines of code, number of commented lines, the number of directories, the number of functions, statements and complexity for each release of each one of our projects. With the results we plot a graph to analyze the growth distribution of our select metrics and draw conclusions. 

\vspace{1 mm}
\noindent{\textbf{Results:}} Table \ref{tab:evolution_overview} shows the difference each project has in terms of metrics in comparison of first and last release.
As we can observe Brackets' first release contains 6271 lines of code but it ends up with 266801 lines of code in the last release. Bracket's last release is the most biggest evolved project with in four years of its existence. It started with 14 directories while in the last release it contains 179 directories. Brackets has 60 releases during its lifetime with 15871 commits. On the other hand NPM with 298 releases is the project with most releases in our dataset. It starts with 1979 lines of code where the last release exceeds to 19837 lines of code with 2231 functions added since the first release.
\todo{create plot to put here}

\begin{table*}[!hbt]
	\begin{center}
		\caption{Release details from each analyzed project}
		\label{tab:evolution_overview}
		\begin{tabular}{l l| c c c c c c}
			\toprule
			\textbf{Project}  & \textbf{Release} & \textbf{LOC} & \textbf{Commented lines} & \textbf{Directories} & \textbf{Functions} & \textbf{Statements} & \textbf{Complexity}\\ \midrule              
			\multirow{2}*{Coffeescript}& First  0.6.1                   &           4693 &           836 &           3 &       915 &       5916 &       2958\\
			& Last   1.9.0                   &           7723 &           124 &           6 &      1262 &       6687 &       5251\\ \midrule
			\multirow{2}*{Less.js     }& First  v1.0                    &           1269 &           279 &           6 &       179 &        818 &        634\\
			& Last   v2.3.1                  &          18585 &          2085 &          18 &      2605 &      13609 &       9414\\ \midrule
			\multirow{2}*{Npm         }& First  v0.0.7                  &           1979 &           259 &           3 &       238 &       1234 &        780\\
			& Last   v2.7.4                  &          19837 &          1060 &          34 &      2469 &      12004 &       5918\\ \midrule
			\multirow{2}*{Mongoose    }& First  0.0.1                   &            554 &            12 &           6 &       100 &        373 &        217\\
			& Last   4.0.1                   &          41844 &          5710 &          12 &      5749 &      27722 &       9373\\ \midrule
			\multirow{2}*{Underscore  }& First  1.0.3                   &           1127 &           154 &           2 &       376 &       1317 &        738\\
			& Last   1.8.0                   &           3918 &           385 &           2 &       849 &       3719 &       1650\\ \midrule
			\multirow{2}*{Node-mysql  }& First  v0.1.0                  &           2431 &            55 &           4 &       203 &       1876 &        435\\
			& Last   v2.6.0                  &          10701 &           430 &          16 &      1046 &       8044 &       2010\\ \midrule
			\multirow{2}*{Q           }& First  v0.1.0                  &            188 &            95 &           1 &        35 &        113 &         64\\
			& Last   v1.1.2                  &           6670 &          1149 &           6 &      1396 &       4118 &       2182\\ \midrule
			\multirow{2}*{Request     }& First  v1.2.0                  &            312 &            12 &           2 &        24 &        214 &        112\\
			& Last   v2.54.0                 &           7839 &           333 &           5 &       958 &       4086 &       1713\\ \midrule
			\multirow{2}*{Ember.js    }& First  sc-v2.0.beta.1          &          28582 &         10006 &          67 &      3879 &      21402 &       9743\\
			& Last   v1.11.0-beta.5          &          65548 &         14698 &         115 &      9323 &      38894 &      13878\\ \midrule
			\multirow{2}*{Source-map  }& First  0.1.0                   &           1214 &           855 &           5 &       124 &        669 &        271\\
			& Last   0.4.1                   &           4485 &           791 &           7 &       322 &       2362 &        757\\ \midrule
			\multirow{2}*{Bootstrap   }& First  v1.3.0                  &            886 &           105 &           2 &       130 &        510 &        261\\
			& Last   v3.3.2                  &           6834 &           358 &           5 &       980 &       4532 &       2337\\ \midrule
			\multirow{2}*{Mocha       }& First  0.0.1-alpha1            &           1185 &           284 &           6 &       226 &        703 &        309\\
			& Last   2.2.0                   &           9931 &          2229 &          21 &      1750 &       6586 &       3197\\ \midrule
			\multirow{2}*{Brackets    }& First  sprint-1                &           6271 &          1198 &          14 &      1305 &       4234 &       2183\\
			& Last   release-1.2-prerelease1 &         266801 &         63923 &         179 &     27845 &     148274 &      82848\\ \midrule
			\multirow{2}*{Bower       }& First  v0.1.1                  &           1149 &           117 &           9 &       212 &        856 &        423\\
			& Last   v1.4.0                  &          15802 &          1439 &          17 &      2454 &       8641 &       3646\\ \midrule
			\multirow{2}*{Grunt       }& First  v0.4.0                  &           3972 &           682 &          11 &       492 &       2482 &        886\\
			& Last   v0.4.4                  &           4010 &           663 &          12 &       498 &       2447 &        874\\ \bottomrule
		\end{tabular}
	\end{center}
\end{table*}

\vspace{3 mm}
\noindent{\rqii}
\vspace{3 mm}

\noindent{\textbf{Motivation:}} It is common to use entire JavaScript libraries inside JavaScript projects, and changes in these libraries can impact your project. Therefore, it is important to understand how often new releases are shipped so you can plan ahead how to use or upgrade these libraries. In the same way, understanding how often third party APIs changes can shape the way you utilize them in your project. 

\vspace{1 mm}
\noindent{\textbf{Approach:}} First, we calculate the average number of releases. We count the number of all versions released and divide it by the number of months between the first release and the last release of each project. Second, we analyze the distribution of the releases. We group the releases in intervals of three months each, and we plot the results to interpret the graph. Third, we use function density to explain the state of public API over time. We calculate the function density in intervals of three months for each project. Then we plot the results for analysis. The formula to calculate function density is the following:

\begin{center}
	$Function Density=  \frac{\left ( \frac{Number Of Functions}{Lines Of Code} \right)}{Number Of Releases}\ast 100$
\end{center}

\vspace{1 mm}
\noindent{\textbf{Results:}} We present the average of number of releases in table \ref{tab:average_release} in general JavaScript projects has a fast passed development with a high number of releases. Npm is the project with the highest average of releases per month 5.13, and the lowest average belongs to Grunt 0.41. 

We present how the releases are distribute in figure \ref{fig:release_density}. Based on the results we find that Mongoose has a higher concentration of releases in the beginning of its development (2012) reaching the peak of 24 releases in three months. Later in its life-cycle (2014) the number of releases drops to 4 and currently (2015) the number of releases increased to 10. We can infer that in the beginning of development more releases are necessary due to the number of changes and the immaturity of the project. Bootstrap in the other hand shows lower variation in the number of releases over time. In general is not possible to say that the distribution of the releases are evenly distributed over time. 

\todo{review}
We depict the function density of the five chosen projects that we found more interesting because of which shows different trends. We calculated the function density during time periods of three months.
Figure \ref{fig:function_density} indicates evolution of our five selected projects. Mongoose starts at 2010 with almost 17\% function density and it has faced a decreasing trend during then. It indicates that after a while developers decided to refactor code to have  excessive use of functions removed. Moreover this project from mid 2012 to 2015 shows interesting phenomena. The function density keeps constant rate, but due to the nature of Mongoose project which is a object modeling library for MongoDB there were no any changes in their APIs after stability. Generally, APIs exposed by function declarations and function expressions in JavaScript. As a result the API keeps unchanged for a long period. 

\par
Like Mongoose we have the same trend for Brackets which after a dramatic decline during 2012 to 2013 the function density becomes unchanged for two years. These trends, again revealed the fact that projects become mature and established and they do not need to change their APIs while they are stable. One might argue these project would be stop developing and the behavior is because they are in maintenance phase or they are becoming legacy software. But We have numerous release followed which we cannot say these projects are abandoned.  

\begin{table}[!hbt]
    \begin{center}
        \caption{Average releases per month}
        \label{tab:average_release}
        \begin{tabular}{l| c c c}
           \toprule
           \textbf{Project} & \textbf{Age in months} & \textbf{\# of releases} & \textbf{Average}\\ \midrule
           Npm          &    58 &  298 &   5.13 \\ 
           Mongoose     &    58 &  197 &   3.39 \\
           Bower        &    30 &   65 &   2.16 \\
           Mocha        &    39 &   80 &   2.05 \\
           Ember.js     &    44 &   87 &   1.97 \\
           Brackets     &    37 &   60 &   1.62 \\
           Q            &    50 &   48 &   0.96 \\
           Source-map   &    41 &   29 &   0.70 \\
           Node-mysql   &    55 &   38 &   0.69 \\
           Bootstrap    &    40 &   27 &   0.67 \\
           Less.js      &    57 &   38 &  0.66 \\
           Request      &    49 &   32 &  0.65 \\
           Coffeescript &    57 &   31 &  0.54 \\
           Underscore   &    56 &   30 &  0.53 \\
           Grunt        &    12 &    5 &  0.41 \\ \bottomrule
      \end{tabular}
    \end{center}
\end{table}

% commented formula
% \begin{center}
% 	$Release Density= \frac{\sum Releases}{3 Months}$
% \end{center}

% release density 
\begin{figure}[thb!]
	\caption{Release density evolution}
	\label{fig:release_density}
	\includegraphics[width=90mm,scale=0.5]{figures/release_density}
	\includegraphics[width=90mm,scale=0.5]{figures/release_density_2}
	\includegraphics[width=90mm,scale=0.5]{figures/release_density_3}
\end{figure}

% Function density 
 \begin{figure}[thb!]
 	\caption{Function density evolution}
 	\label{fig:function_density}
 	\includegraphics[width=90mm,scale=0.5]{figures/function_density}
 	\includegraphics[width=90mm,scale=0.5]{figures/function_density_2}
 	\includegraphics[width=90mm,scale=0.5]{figures/function_density_3}
 \end{figure}

\vspace{3 mm}
\noindent{\rqiii}
\vspace{3 mm}

\par For number of developers we tried to gather data based on email of authors that commit code, however we found out that authors can have multiple emails register to one login id. To overcome this problem we tried to count the number of developers based on the Github usernames. 

\par
Having comments as one of the artifacts that can be a sign of improvement in the readability of source code we examined it as metric of evolution.
Brackets has the most comment lines with having 63923 comment lines followed by Ember.js with 14698 comment lines. The less documented project is CoffeeScript with having only 124 lines of documentation. What makes the result interesting, is the fact that CoffeeScript starts with 836 lines of comments in the first release but in the last release we have only 124 lines of code. CoffeeScript and Grunt are the only projects that adapt this pattern, while all the other projects show growing trend in number of comments during evolution.  

The next characteristics we measure in evolution is to measure how much a project is community driven. We count number of developers having at least one commit in each period between releases. Figure \ref{fig:number_of_developers} show the number of developers in different time slots. Request and Bootstrap are among projects with over 80 and 100 developers in specific peaks. Despite the fact that Underscore has fluctuated so much, at recent releases it shows that it gains some more attention and it becomes more community driven in terms of number of volunteer contributors. Project \textit{q} takes less attention as compare to other projects. It has never had more than 20 developers since its beginning in mid 2010.
\par
Request project had one drastic fell down in mid 2013. By exploring in more details, we found that they have two releases which are closed to each other. The first one has more than 80 developers while the other has less 10 developers. This indicates they found some major issues in former release and few core developers instantly fix them and another release followed by previous one. One interesting phenomena that we observe is the number of released skewed to the right of each graph. Best practices in software engineering suggest having iterative and incremental development and we found out developers tend to have more smaller release in time rather than having one huge release every 6 months.  

% developers per release 
\begin{figure}[thb!]
	\caption{Number of developers per release}
	\label{fig:number_of_developers}
	\includegraphics[width=90mm,scale=0.5]{figures/number_of_developers}
	\includegraphics[width=90mm,scale=0.5]{figures/number_of_developers_2}
	\includegraphics[width=90mm,scale=0.5]{figures/number_of_developers_3}
\end{figure}


\ref{fig:number_of_issues}
Having a software without bug means that software is never used by users. When it comes to bugs, we can find where and in what extent the software's problem are saturated. The module or component with more issues is an indication of complex. Taking a look in each project and density of issues during releases get us an insight into on how quality JavaScript projects are developed. Figure \ref{fig:number_of_issues} shows numbers of issues introduced between releases.

\begin{figure}[thb!]
	\caption{Number of issues introduced in each release}
	\label{fig:number_of_issues}
	\includegraphics[width=90mm,scale=0.5]{figures/issues_per_release}
	\includegraphics[width=90mm,scale=0.5]{figures/issues_per_release_2}
	\includegraphics[width=90mm,scale=0.5]{figures/issues_per_release_3}
\end{figure}

 Figure \ref{fig:avg_issue_fix} shows the average time it takes to get project fixed in all JavaScript projects.
 
\begin{figure}[thb!]
	\caption{Average time to fix issues}
	\label{fig:avg_issue_fix}
	\includegraphics[width=90mm,scale=0.5]{figures/average_days_to_fix_bugs}

\end{figure}


Figure \ref{fig:number_of_issues} shows Bootstrap in special releases had around 1000 issues caught by testers and developers. Looking back to the nature of this project, it is obvious such a huge library which provides lots of functionalities would probably have numerous users whom they can report bugs. Moreover for the release in the mid of 2013 we can see the distance between the release and previous ones was significant. By looking at the data we found out that release was one of the major releases that Bootstrap had before. 

\begin{table}[!hbt]
	\begin{center}
		\caption{Average time to fix a bug}
		\label{tab:average_time_bugfix}
		\begin{tabular}{l| c }
			\toprule
			\textbf{Project}  & \textbf{Average bug fix in days} \\ \midrule              
			Coffeescript    & 46  \\
			Less.js         & 113 \\
			Npm             & 58  \\
			Mongoose        & 44  \\
			Underscore      & 14  \\
			Node-mysql      & 47  \\
			Q               & 30  \\
			Request         & 105 \\
			Ember.js        & 22  \\
			Source-map      & 17  \\
			Bootstrap       & 11  \\
			Mocha           & 83  \\
			Brackets        & 26  \\
			Bower           & 30  \\
			Grunt           & 28  \\  \bottomrule
		\end{tabular}
	\end{center}
\end{table}

To study more about these different projects we conduct another analysis to find the average time it takes to fix a bug that reported after releases. We use Github issue tracker to extract all information related to bugs and then we grouped them by releases. Then the average time it takes since the bug was created to the date it has closed is calculated. Table \ref{tab:average_time_bugfix} shows the result that the average time in days that it takes to fix bugs in each of selected projects.  
\par 
What interesting fact we found is that Bootstrap has the least number of days taken to fix a bug. Considering number of developers Bootstrap has based on Figure \ref{fig:number_of_developers} we found that this project has one of the most active communities among other projects. While Less.js is a project that on average it takes 113 days to get a bug fixed with fewer developers. 
\par
\textit{Underscore} has a good response and fixed rate among projects. While we are comparing the rate of bug fixing, we observe that Underscore is also one of the projects that has large number of developers contributing to the project. By having deeper observation into the data, we found out that number of developers are positively correlated to the time it takes to fix and close a bug.

\par For project \textit{q} it takes on average 30 days to fix bugs. By looking at number of developers we found out it has comparably less developers as compare to other projects but we have to take into account the fact of projects size and scale are in different order. A project that has fewer functionality might has fewer bugs and as a consequence less time would it take to fix a bug.

\vspace{5 mm}
\noindent{\rqiv}
\vspace{5 mm}

\subsection{JavaScript Rule violations}
We use SonarQube to use its various rules for finding code smells in JavaScript projects. Utilizing SonarQube give us the ability to extract code smells with different severity. To explain it more we list following rules as an example for issues with different severity:

	\begin{table*}[!hbt]
		\begin{center}
			\caption{Bad smells breakdown from each analyzed project}
			\label{tab:bad_smell_evolution_overview}
			\begin{tabular}{l l| c c c c c }
				\toprule
				\textbf{Project}  & \textbf{Release} & \textbf{Blocker smells} & \textbf{Critical smells} & \textbf{Major smells} & \textbf{Minor smells} & \textbf{Info smells} \\ \midrule              
				\multirow{2}*{Coffeescript}& First  0.6.1                   &           0 &           2 &          1692 &       31 &        0 \\
				& Last   1.9.0                   &           0 &           0 &          3528 &       48 &        0 \\ \midrule
				\multirow{2}*{Less.js     }& First  v1.0                    &           0 &           0 &           336 &       24 &        0 \\
				& Last   v2.3.1                  &           3 &           0 &          4285 &      317 &       26 \\ \midrule
				\multirow{2}*{Npm         }& First  v0.0.7                  &           0 &          34 &          1300 &       10 &       11 \\
				& Last   v2.7.4                  &           0 &          18 &         12535 &       49 &        8 \\ \midrule
				\multirow{2}*{Mongoose    }& First  0.0.1                   &           1 &           1 &           316 &        4 &        0 \\
				& Last   4.0.1                   &          22 &          12 &          4773 &      161 &        8 \\ \midrule
				\multirow{2}*{Underscore  }& First  1.0.3                   &           0 &           0 &           465 &       10 &        0 \\
				& Last   1.8.0                   &           1 &           0 &          1000 &       19 &        0 \\ \midrule
				\multirow{2}*{Node-mysql  }& First  v0.1.0                  &           2 &           0 &           230 &       10 &        2 \\
				& Last   v2.6.0                  &          27 &           2 &           479 &       72 &        4 \\ \midrule
				\multirow{2}*{Q           }& First  v0.1.0                  &           0 &           0 &            15 &        1 &        1 \\
				& Last   v1.1.2                  &           0 &           1 &           319 &       41 &       18 \\ \midrule
				\multirow{2}*{Request     }& First  v1.2.0                  &           0 &           0 &           115 &        2 &        0 \\
				& Last   v2.54.0                 &           0 &           0 &          3798 &       20 &       71 \\ \midrule
				\multirow{2}*{Ember.js    }& First  sc-v2.0.beta.1          &           0 &           1 &          8469 &      586 &       24 \\
				& Last   v1.11.0-beta.5          &           0 &           0 &          1489 &      343 &       62 \\ \midrule
				\multirow{2}*{Source-map  }& First  0.1.0                   &           1 &           1 &            91 &        5 &        1 \\
				& Last   0.4.1                   &           6 &           1 &           268 &        7 &        1 \\ \midrule
				\multirow{2}*{Bootstrap   }& First  v1.3.0                  &           0 &           0 &           523 &        7 &        0 \\
				& Last   v3.3.2                  &           0 &           0 &          4777 &       49 &       55 \\ \midrule
				\multirow{2}*{Mocha       }& First  0.0.1-alpha1            &           0 &           0 &           233 &        2 &        2 \\
				& Last   2.2.0                   &           0 &           0 &          1675 &       76 &       10 \\ \midrule
				\multirow{2}*{Brackets    }& First  sprint-1                &           0 &          12 &          1167 &       35 &       70 \\
				& Last   release-1.2-prerelease1 &           2 &           3 &         40985 &     5248 &     1879 \\ \midrule
				\multirow{2}*{Bower       }& First  v0.1.1                  &           0 &           0 &           169 &        1 &        0 \\
				& Last   v1.4.0                  &           7 &         117 &           226 &       44 &       18 \\ \midrule
				\multirow{2}*{Grunt       }& First  v0.4.0                  &         123 &           0 &           137 &       11 &       28 \\
				& Last   v0.4.4                  &         142 &           0 &           155 &       10 &       26 \\ \bottomrule
			\end{tabular}
		\end{center}
	\end{table*}

\begin{itemize}
	\item \textbf{"===" and "!=="}:
The == and != operators do type coercion before comparing values. This is bad because it can mask type errors. For example, it evaluates \escape{t} \escape{r} \escape{n} == 0 as true.

It is best to always use the side-effect-less === and !== operators instead. This issue is considered as \textit{Major} issue.

		\begin{lstlisting}[caption=Noncompliant Code Example]
		if (var == 'howdy') {...} // Noncompliant
		\end{lstlisting}
		
		\begin{lstlisting}[caption=Compliant Solution]
		if (var === 'howdy') {...}
		\end{lstlisting}


	\item \textbf{\textit{continue} should not be used}
	continue is an unstructured control flow statement. It makes code less testable, less readable and less maintainable. Structured control flow statements such as if should be used instead. We considered this issue as \textit{Critical} issue.
	
		
		\begin{lstlisting}[caption=Noncompliant Code Example]
		for (i = 0; i < 10; i++) {
		if (i == 5) {
		continue;  /* Non-Compliant */
		}
		alert("i = " + i);
		}
		\end{lstlisting}
		
		\begin{lstlisting}[caption=Compliant Solution]
		for (i = 0; i < 10; i++) {
		if (i != 5) {  /* Compliant */
		alert("i = " + i);
		}
		}
		\end{lstlisting}
	
	
	
	
	\item \textbf{ \textit{switch} statements should end with a \textit{default} clause}
	he requirement for a final default clause is defensive programming. The clause should either take appropriate action or contain a suitable comment as to why no action is taken. It is a \textit{Major} issue.
	

		
		\begin{lstlisting}[caption=Noncompliant Code Example]
		switch (param) {  //missing default clause
		case 0:
		 doSomething();
		break;
		case 1:
		 doSomethingElse();
		 break;
		}
		
		switch (param) {
		 default: // default clause should be the last one
		 error();
		 break;
		case 0:
		 doSomething();
		 break;
		case 1:
		 doSomethingElse();
		 break;
		}
		\end{lstlisting}
		
		\begin{lstlisting}[caption=Compliant Solution]
		switch (param) {
		case 0:
		 doSomething();
		 break;
		case 1:
		 doSomethingElse();
		 break;
		default:
		 error();
		 break;
		}
		\end{lstlisting}	
	
	
	\item \textbf{\textit{with} statements should not be used}
	The use of the with keyword produces an error in JavaScript strict mode code. However, that is not the worst that can be said against with.
	
	Using with allows a short-hand access to an object's properties - assuming they are already set. But use with to access some property not already set in the object, and suddenly you are catapulted out of the object scope and into the global scope, creating or overwriting variables there. Since the effects of with are entirely dependent on the object passed to it, with can be dangerously unpredictable, and should never be used. It is a major issue.
	
		\begin{lstlisting}[caption=Noncompliant Code Example]
	var x = 'a';
	
	var foo = {
		y: 1
	}
	
	with (foo) {  // Noncompliant
		y = 4;  // updates foo.x
		x = 3;  // does NOT add a foo.x property; updates x var in outer scope
	}
	print(foo.x + " " + x); // shows: undefined 3
		\end{lstlisting}
		
			\begin{lstlisting}[caption=Compliant Solution]
			var x = 'a';
			
			var foo = {
			 y: 1
			}
			foo.y = 4;
			foo.x = 3;
			
			print(foo.x + " " + x); // shows: 3 a
			\end{lstlisting}
			
		
	\item \textbf{Statements should be on separate lines}
	For better readability, do not put more than one statement on a single line. It is a minor issue.
	
	\begin{lstlisting}[caption=Noncompliant Code Example]
	if(someCondition) doSomething();
	\end{lstlisting}
	
	\begin{lstlisting}[caption=Compliant Solution]
if(someCondition) {
doSomething();
}
	\end{lstlisting}
	
	

\end{itemize}


Table \ref{tab:bad_smell_evolution_overview} shows the evolution of bad smells with different severities. Most of projects has no or few Blocker smells.  Except project \textit{Grunt} we observed that all project start with zero or less than two blocker bugs. Project \textit{node-mysql} starts with 2 blocker issues and it had 27 in the last release. The weight of Major smells revealed that lacking of a good static code checker and compiler let developers to have more and more rule violations with our any warning from absent compiler in JavaScript. Although JavaScript developers use source code inspection tools such as \textit{JSLint} and \textit{JSHint} we still see rule violation in such a huge extent as compare to Java projects. Statically typed languages such as Java would not let developer to have piece of code that violates type casting rules or having an undeclared variable.
	
	
	

\section{Threats to validity}
\label{sec:threats}
\par
We found one project was migrated from SVN before the first commit in git. Github keeps track of tags that were imported from another source control but based on unknown reasons it does not import commits. So we faced with a scenario that we tag with date before first commit initiated on Github. 


\par
Moreover, all of the projects that we select are either server side codes or they are client side libraries for enhancement of development in client side web applications. But we did not take it into account to have JavaScript codes that are part of the web application. The results from client site codes which are not part of client libraries can reflect different behavior than what we observe on our selected dataset.


\section{Related Work}
\label{sec:related_work}
Kyriakakis \emph{et al.} \cite{Kyriakakis2014ICMSE} Analyzed the evolution of PHP projects over time. They work contributes to the evolution scenario because script languages as PHP are often labeled as inappropriate for big projects, under the claim that these languages are not maintainable. During their experiment they analyzed the evolution of five open source projects with a long development history. They analyzed the amount of unused code, the removal of functions, the use of libraries, the stability of the interfaces, the migration to object-orientation and the evolution of complexity. Based on their results they concluded that these projects are submitted to organized changes and that despite the fact of the language adaptive, and perfective maintenance does stills takes place in the projects. They also find that all the projects are gradually migrating to object-orientation fact that indicates planed maintenance as this feature was added in the language after the start of the development of all projects. 

Similarly to this work we pretend to understand how the JavaScript language evolves over time, collaborating with the script languages evolution landscape. In addition to that, we search for possible applications of lessons learned analyzing the evolution of static languages as Java. Finding similar evolution patterns between script and static languages will lead us to best practices that can be applied in both contexts. 

Godfrey \emph{et al.} \cite{Godfrey2000ICMS} in their work on software evolution they analyzed how large open source projects evolves over time. Until their research, most work in software evolution was related with ``in home'' solutions. They collected and analyzed 6 years of development data from Linux kernel a large open source project with over two million of lines of code in the latest version analyzed at the time. They measured the length of the each full distribution, the lines of code - considering commented and blanks lines and then ignoring them as well - , the number of global functions, variables and macros. They found that Linux kernel growth has been super-linear despite the fact of its large size, the collaboration of several volunteers developers that are scattered around the world and that previous research in the area had found that growth of large software systems tends to slow down as the system becomes larger. They also found several important facts particular to the Linux Kernel system as although the source tree for linux is rather large more than half of the code belongs to devices drivers. 

As this previous work we analyze the evolution of open source projects, but in particular we focus our analyzes in several small to mid sized  open sources projects written in two different languages, one dynamic (JavaScript) and the other one static (Java). We conduct an empirical study to analyze two main factors, first how JavaScript projects evolves over time and second, what are the lessons that more studied static languages can be applied as a good practice in JavaScript.

%\everton{ use this paragraph in the related work for JavaScript. Note that it was just measuring metrics not evolution. but we have to mention it: 
%	Starts here:
%	Previous works on JavaScript calculate trivial metrics such as number of attributes (NOA), number of methods (NOM), Depth of Inheritance Tree (DIT) and number of children. Possible actions can be taken for: (1) recalculating and evaluating those metrics, (2) calculating advanced metric complexity of a function or file. }


\section{Future Work}
\label{sec:future}
Our approach followed an automatic mechanism to extract, refine and analyze source codes without human intervention. By utilizing this approach we are able to have a big data set helping us to achieve more precise result. Certainly we want to have equal number of Java projects to be able comparing evolution of these two different language. SonarQube is capable of measuring metrics regardless of target language. That study would certainly reveal hidden corners that we did not catch by studying only JavaScript projects and if the outcome shows similarities, making decision for stakeholders, project owners and developers would be easier to have best practices from Java which established for more than a decade in an enterprise level applications.
JavaScript lacks standard language specification for defining classes. Classes are used to achieve re-usability in JavaScript. We developed a tool for finding these classes using static analysis. For another study we extract object oriented metrics from well-known JavaScript such as McCabe cyclomatic complexity, coupling between objects \cite{Briand-Coupling} and cohesion\cite{Briand-Cohesion}. These metrics can help us to better understand JavaScript projects and also we can compare these metrics with values extracted from Java projects correspondingly. 
We plan to extensively study differences and similarities between Java and JavaScript project to gain deeper insight into these two different world. To achieve our goal we have been considering every corner cases we would face in our study like lacking of explicit class definition in JavaScript and different style of defining namespaces in JavaScript. Namespaces are the only way JavaScript developers rely to achieve higher degree of granularity.

\section{Conclusion}
\label{sec:conclusion}
We conduct an empirical study in the evolution of JavaScript projects. We analyzed 15 open source projects from different domains of application. In general each one of the projects has more than 5 years of development history allowing us to explore over 57 years of software evolution.

First we analyze how JavaScript projects grows over time. We measured all the projects by the following attributes: lines of code, number of commented lines, number of directories, number of functions, statements and complexity. We found that all the projects increased with very pace , and some times with huge leaps in the number of lines of code from one version to another. This is due mostly to the dynamic environment and the community that surrounds JavaScript projects (e.g., A popular project will draw many developers to join the development) in addition to the customization of open source libraries that becomes part of the project own source code. 

We found that developers are worried about evolving the project in a maintainable manner as we can see an increase in the commented lines and in the number of directories as the time goes by. We infer that the growth of directories is a signal of modularization in JavaScript projects, as the language does not support directly the creation of namespaces. Other than that, the grow of commented lines is a signal of code documentation in JavaScript. However, more research is needed to collect more evidence of these behavior.

Second, we analyze the average time to ship new releases and the changes in the public APIs. We found that versions tend to be released fast in most of the projects. The most active project was NPM with an average of 5.13 releases per month, and the lowest average belongs to Grunt with 0.41 releases a month. Regarding the changes in the API, we found that the analyzed projects tend to have changes in their APIs until they reach a maturity point and become stable during several releases. The time to reach maturity varies from project to project.

Third, we found that the JavaScript community has many independent contributors beside the core-developers, although this characteristic varies with the size and domain of the project. In terms of time to fix a bug we found that for 86\% of the analyzed projects it takes less than 60 days to fix a bug. The fastest project to fix a bug was Bootstrap with an average of 11 days and the slowest is project Less with an average of 113 days.  

Fourth, we analyzed how common are the anti-patterns in JavaScript software and how severe they are. We found that anti-patterns appears in JavaScript projects with frequency. However, they are not the most severe kind of faults. The concentration of these anti-patterns are mostly in two categories, major and minor. The tendency of these anti-patterns over time is increasing, but we found an exception in Ember.js where this trend shows decreasing.   

\bibliographystyle{IEEEtran}
\bibliography{design_td}

\end{document}