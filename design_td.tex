\documentclass[conference]{IEEEtran}

\usepackage{amssymb,amsmath}
\usepackage{wrapfig}
\usepackage{multirow}
\usepackage{graphicx}
\usepackage{algorithm}
\usepackage{algorithmic}
\usepackage{times}
\usepackage{cite}
\usepackage{url}
\usepackage{booktabs}
\usepackage{subfigure}
\usepackage{fancybox}
\usepackage{color}
\usepackage{array}
\usepackage{subfigure}
\usepackage{balance}
\usepackage{epstopdf}
\usepackage{array}


\newcommand{\shahriar}[1]{\textcolor{red}{{\it [Shahriar: #1]}}}
\newcommand{\everton}[1]{\textcolor{blue}{{\it [Everton: #1]}}}
\newcommand{\todo}[1]{\colorbox{yellow}{\textbf{[#1]}}}



\newcommand{\conclusionbox}[1]{%
	\vspace{2mm}
	\framebox[0.45\textwidth][c]{%
		\parbox[b]{0.42\textwidth}{%
			{\it #1}
		}
	}
	\vspace{2mm}
}

\newcommand{\rqi}{\textbf{}}
\newcommand{\rqii}{\textbf{}}
\newcommand{\rqiii}{\textbf{}}

\begin{document}
	\title{An Empirical Study on Evolution of Open Source JavaScript pr Projects}
	
	\author{\IEEEauthorblockN{Everton da S. Maldonado and Shahriar Rostami Dovom}
		
		\IEEEauthorblockA{Department of Computer Science and Software Engineering\\Concordia University,
			Montreal, Canada\\
			\url{everton.maldonado@gmail.com}, \url{shahriar.rostami@gmail.com}}}
	
	\maketitle
	
	\begin{abstract}
		\shahriar{hello maldonado}
		
	\end{abstract}
	
	\IEEEpeerreviewmaketitle
	
	\section{Introduction}
	\label{sec:introduction}
	
	\section{Related Work}
	\label{sec:related_work}
	
	% \begin{figure*}[thb!]
	% 	\caption{Approach overview}
	% 	\centering
	% 	\label{fig:approach}
	% 	\includegraphics[width=1\textwidth]{figures/Approach2}
	% \end{figure*}
	
	
	\section{Approach}
	\label{sec:approach}
We decide to examine the evolution of five JavaScript server-side packages known as Node Packages and five Java projects to compare how might the similarities and differences on how these two popular programming language can affect evolution of software systems. 
	
	\section{Research Questions and techniques}
	Metrics that we are going to use: 
	1- Lines of Code,
	
	\shahriar{Lehman suggests using the number of “modules” as the best way to measure the size of a large software system \cite{637156}. However, we decided to use the number of uncommented lines of code (“uncommented LOC”) for most of our mea- surements for several reasons. First, as discussed below, we found that total system uncommented LOC seemed to grow at roughly the same rate as the number of source files; how- ever, as shown by the difference between average and me- dian file size below, there was great variation in file size in some parts of the system. We decided, therefore, that using number of source files would mean losing some of the full story of the evolution of Linux, especially at the subsystem level. \cite{883030}
	}
	2- Directory Structure?
	
	
	
	shitoriiiiiiiiiiiiiiiiiiii
	
	
	\label{sec:rq}
	
	\section{Milestones}
	
	\bibliographystyle{IEEEtran}
	\bibliography{design_td}
\end{document}
