We conduct a empirical study in the evolution of JavaScript projects. We analyzed 15 open source projects from different domains of application. In general each one of the projects has more than 5 years of development history allowing us to explore over 57 years of software evolution.

First we analyze how JavaScript projects grows over time. We measured all the projects by the following attributes: lines of code, number of commented lines, the number of directories, the number of functions, statements and complexity. We found that all the projects increased in a very fast way, and some times with huge leaps in the number of lines of code from one version to another. This is due mostly to the dynamic environment and the community that surrounds JavaScript projects (e.g., A popular project will draw many developers to join the development) in addition to the customization  of open source libraries that becomes part of the project own source code. 

We found that developers are worried about evolving the project in a maintainable manner as we can see an increase in the commented lines and in the number of directories as the time goes by. We infer that the growth of directories is a signal of componentization in JavaScript projects, as the language does not support directly the creation of namespaces. Other than that, the grow of commented lines is a signal of code documentation in JavaScript. However, more research is needed to collect more evidence of these behavior.

Second, we analyze the average time to ship new releases and the changes in the public APIs. We found that releases tend to be released fast in most of the projects. The most active project was Npm with an average of 5.13 releases per month, and the lowest average belongs to Grunt with 0.41 releases a month. Regarding the changes in the API, we found that the analyzed projects tend to have changes in their APIs until they reach a maturity point and become stables during several releases. The time to reach maturity varies from project to project.

Third, we found that the JavaScript community has many independent contributors beside the core-developers, although this characteristic varies with the size and domain of the project. In terms of time to fix a bug we found that for 86\% of the analyzed projects it takes less than 60 days to fix a bug. The fastest project to fix a bug was Bootstrap with an average of 11 days and the slowest is Less with an average of 113 days.  

Fourth, we analyzed how common are the anti-patterns in JavaScript software and how severe they are. We found that anti-patterns appears in JavaScript projects with frequency. However, they are not the most severe kind of faults. The concentration of these anti-patterns are mostly in two categories, major and minor. The tendency over time is to this number increase, but we found an exception in Ember.js where this number is continuously decreasing.   