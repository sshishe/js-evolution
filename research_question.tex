Lehman suggests using the number of “modules” as the best way to measure the size of a large software system \cite{637156}. However, we decided to use the number of uncommented lines of code (“uncommented LOC”) like the way Godfrey et al \cite{883030} did the evolution study on Linux Kernel. On the other hand we measure the comment lines and the ratio of comments to lines of codes, and based on that we can infer how much developers tend to put comments within their codes. We have to consider hidden corners that can mislead results, for example descriptive comments are totally different to the lines of codes that got commented because of refactoring or changes which consider as light-weight code smells within the code.
We are going to answer following research questions:


\rqi

\noindent{\textbf{Motivation:}}

\noindent{\textbf{Approach:}}

\rqii

\noindent{\textbf{Motivation:}}

\noindent{\textbf{Approach:}}

\rqiii

\noindent{\textbf{Motivation:}}

\noindent{\textbf{Approach:}}


\begin{itemize}
	\item How and in what extent can we measure object oriented metrics in JavaScript?
	\item  By admitting the fact that JavaScript developers emulate object oriented style programming, we can empirically compare the differences between this language and mainstream programming languages.
	\item Studying the evolution of JavaScript libraries in terms of software metrics to find out how mature developers are writing code in terms of obeying object oriented best practices.
\end{itemize}
	

	
	

