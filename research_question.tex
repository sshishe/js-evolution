
We are going to answer following research questions:


\rqi

\noindent{\textbf{Motivation:}}
We want to answer this question to explore how a statically strong typed language and a semi-functional, object oriented language with dynamic capability can affect projects quality. We will measure different metrics that we introduce in section \textit{Approach} such as line numbers, lines and number of function. We will also measure how developers use re-usable patterns such as class declaration and inheritance in their codes. Despite the fact that JavaScript developers use class declarations and as we mentioned there is no direct support in the language specification, we believe developers tend to mimic class declarations more and more in their code. If this hypothesis grounds, then we can infer what causes lead to this phenomena that developers of JavaScript change the way they are programming.
\par
\noindent{\textbf{Approach:}}
Having five versions of each software that mentioned in \ref{eval_table}, we can justify how and in what extent the software evolves in terms of those metrics. 

\par
\rqii

\noindent{\textbf{Motivation:}}
		 The term technical debt is used to express faults and non-optimal solutions made conscientiously in a software project in order to achieve a short-term goal like meeting urgent deadlines or fixing last minutes bugs. The metaphor plays an important role in the communication between developers and managers \cite{kruchten2013MTD}, as the financial model fits well to explain delicate scenarios, that otherwise, could lead to a heavy technical conversations that, most of the time, is proven to be unproductive for both sides.  
		 
		 \par
		 Prior work showed that technical debt is unavoidable \cite{zazworka2013MTD} and that there are different types of technical debt, e.x. defect debt, design debt, testing debt, documentation debt, etc \cite{Seaman2011MMTD} . One of the most impacting types is design debt \cite{zazworka2011MTD}\cite{Fontana2012MTD} (a.k.a Bad Smells). A lot of effort are made to understand and manage TD in different domains as databases schemas \cite{Weber2014MTD} or even in the development community \cite{Tamburri2013CHASE}. However, to the best of our knowledge, very scarce research has been done in JavaScript applications from the technical debt perspective. Thus we collaborate with the technical debt landscape analyzing its behavior in JavaScript applications. 

\noindent{\textbf{Approach:}}

\rqiii

\noindent{\textbf{Motivation:}}

\noindent{\textbf{Approach:}}


\begin{itemize}
	\item How and in what extent can we measure object oriented metrics in JavaScript?
	\item  By admitting the fact that JavaScript developers emulate object oriented style programming, we can empirically compare the differences between this language and mainstream programming languages.
	\item Studying the evolution of JavaScript libraries in terms of software metrics to find out how mature developers are writing code in terms of obeying object oriented best practices.
\end{itemize}
	

	
	

