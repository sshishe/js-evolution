We select a long-lived projects to use in our case study based on the following criteria:
We select projects that are popular between developers and are actively developed for at least three consecutive years.
	We used projects that have their source code and issuer tracker both hosted on famous open source repository, GitHub. This makes it easier for us to collect information about collaboration and to measure popularity of the projects based on information we gather from website. Besides, GitHub provides us an API to extract data from their repositories to co-evaluate with the evolution analysis that we will run. 
	Each project should have at least 20 unique tags on Github.
	Large number of commits was also one of our criterion to select the project.

Based on Samoladas\cite{Samoladas2010SAD}, majority of open source softwares are not developed after a short time period, makes them inappropriate for evolution analysis. Our dataset consist of total 
57 years of history with 1065 releases. 

Our study conducted on projects listed as following:
\begin{itemize}
	\item \textbf{CoffeeScript} is a superset language of JavaScript and it adds syntactic sugar inspired by Python, Ruby and Haskell.
	\item \textbf{less.js} extends CSS with dynamic behavior such as variables, mixins, operations and functions. Less runs on both the server-side and client-side.
	\item \textbf{NPM} is the package manager for JavaScript either in browser or server-side.
	\item \textbf {Mongoose} is an elegant Mongodb object modeling for node.js.
	\item \textbf{Underscore} is a JavaScript library that provides a whole useful functional programming helpers without extending any built-in objects.
	\item \textbf {Node-mysql} is a node.js driver for mysql.
	\item \textbf {q} is a tool for creating and composing asynchronous promises in JavaScript.
	\item \textbf {Request} is a simplified HTTP request client for node.js.
	\item \textbf{Ember.js} is an open-source client-side JavaScript web application framework based on the model-view-controller (MVC) software architectural pattern.
	\item \textbf{Source-Map} is a library to generate and consume the source map.
	\item \textbf{Bootstrap} intuitive, and powerful mobile first front-end framework for faster and easier web development.
	\item \textbf{Mocha} is a feature-rich JavaScript test framework running on node.js and the browser, making asynchronous testing simple.
	\item \textbf{Brackets} is a modern, open source text editor that understands web design (HTML, JavaScript and CSS).
	\item \textbf{Bower} is a package manager for Web (specially client-side).
	\item \textbf{Grunt} a JavaScript task runner integrated with node.js.	
\end{itemize} 
Due to the project's long history we have numerous versions of each project we show some statistics about projects.

\begin{table}[!hbt]
	\begin{center}
		\caption{Release details from each analyzed project}
		\label{tab:release_details}
		\begin{tabular}{l| c c c c c c}
			\toprule
			\textbf{Project} & \textbf{First release} & \textbf{Last release}  & \textbf{\# Releases} & \textbf{\# Commits} \\ \midrule
			Coffeescript & 2010    & 2015  &    31 &      3967 \\
			Less.js      & 2010    & 2015  &    38 &      2428 \\
			Npm          & 2010    & 2015  &   298 &      5509 \\
			Mongoose     & 2010    & 2015  &   197 &      4789 \\
			Underscore   & 2010    & 2015  &    30 &      1985 \\
			Node-mysql   & 2010    & 2015  &    38 &       914 \\
			Q            & 2010    & 2014  &    48 &      3359 \\
			Request      & 2011    & 2015  &    32 &      1644 \\
			Ember.js     & 2011    & 2015  &    87 &      9150 \\
			Source-map   & 2011    & 2015  &    29 &       396 \\
			Bootstrap    & 2011    & 2015  &    27 &     11195 \\
			Mocha        & 2011    & 2015  &    80 &      1702 \\
			Brackets     & 2011    & 2015  &    60 &     15871 \\
			Bower        & 2012    & 2015  &    65 &      1813 \\
			Grunt        & 2013    & 2014  &     5 &      1309 \\ \bottomrule
		\end{tabular}
	\end{center}
\end{table}
