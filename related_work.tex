Kyriakakis \emph{et al.} \cite{Kyriakakis2014ICMSE} Analyzed the evolution of PHP projects over time. They work contributes to the evolution scenario because script languages as PHP are often labeled as inappropriate for big projects, under the claim that these languages are not maintainable. During their experiment they analyzed the evolution of five open source projects with a long development history. They analyzed the amount of unused code, the removal of functions, the use of libraries, the stability of the interfaces, the migration to object-orientation and the evolution of complexity. Based on their results they concluded that these projects are submitted to organized changes and that despite the fact of the language adaptive, and perfective maintenance does stills takes place in the projects. They also find that all the projects are gradually migrating to object-orientation fact that indicates planed maintenance as this feature was added in the language after the start of the development of all projects. 

Similarly to this work we pretend to understand how the JavaScript language evolves over time, collaborating with the script languages evolution landscape. In addition to that, we search for possible applications of lessons learned analyzing the evolution of static languages as Java. Finding similar evolution patterns between script and static languages will lead us to best practices that can be applied in both contexts. 

%\everton{ use this paragraph in the related work for JavaScript. Note that it was just measuring metrics not evolution. but we have to mention it: 
%	Starts here:
%	Previous works on JavaScript calculate trivial metrics such as number of attributes (NOA), number of methods (NOM), Depth of Inheritance Tree (DIT) and number of children. Possible actions can be taken for: (1) recalculating and evaluating those metrics, (2) calculating advanced metric complexity of a function or file. }
