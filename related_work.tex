Kyriakakis \emph{et al.} \cite{Kyriakakis2014ICMSE} Analyzed the evolution of PHP projects over time. They work contributes to the evolution scenario because script languages as PHP are often labeled as inappropriate for big projects, under the claim that these languages are not maintainable. During their experiment they analyzed the evolution of five open source projects with a long development history. They analyzed the amount of unused code, the removal of functions, the use of libraries, the stability of the interfaces, the migration to object-orientation and the evolution of complexity. Based on their results they concluded that these projects are submitted to organized changes and that despite the fact of the language adaptive, and perfective maintenance does stills takes place in the projects. They also find that all the projects are gradually migrating to object-orientation fact that indicates planed maintenance as this feature was added in the language after the start of the development of all projects. 

Similarly to this work we pretend to understand how the JavaScript language evolves over time, collaborating with the script languages evolution landscape. In addition to that, we search for possible applications of lessons learned analyzing the evolution of static languages as Java. Finding similar evolution patterns between script and static languages will lead us to best practices that can be applied in both contexts. 

Godfrey \emph{et al.} \cite{Godfrey2000ICMS} in their work on software evolution they analyzed how large open source projects evolves over time. Until their research, most work in software evolution was related with ``in home'' solutions. They collected and analyzed 6 years of development data from Linux kernel a large open source project with over two million of lines of code in the latest version analyzed at the time. They measured the length of the each full distribution, the lines of code - considering commented and blanks lines and then ignoring them as well - , the number of global functions, variables and macros. They found that Linux kernel growth has been super-linear despite the fact of its large size, the collaboration of several volunteers developers that are scattered around the world and that previous research in the area had found that growth of large software systems tends to slow down as the system becomes larger. They also found several important facts particular to the Linux Kernel system as although the source tree for linux is rather large more than half of the code belongs to devices drivers. 



As this previous work we analyze the evolution of open source projects, but in particular we focus our analyzes in several small to mid sized  open sources projects written in two different languages, one dynamic (JavaScript) and the other one static (Java). We conduct an empirical study to analyze two main factors, first how JavaScript projects evolves over time and second, what are the lessons that more studied static languages can be applied as a good practice in JavaScript.

%\everton{ use this paragraph in the related work for JavaScript. Note that it was just measuring metrics not evolution. but we have to mention it: 
%	Starts here:
%	Previous works on JavaScript calculate trivial metrics such as number of attributes (NOA), number of methods (NOM), Depth of Inheritance Tree (DIT) and number of children. Possible actions can be taken for: (1) recalculating and evaluating those metrics, (2) calculating advanced metric complexity of a function or file. }
